\documentclass[12pt,a4paper]{article}

\usepackage[margin=2.5cm,left=4cm]{geometry}
\usepackage{setspace}
\usepackage{titling}
\usepackage{times}
\usepackage{enumitem}
\usepackage{hyperref}
\usepackage{graphicx}
\usepackage{ifthen}

\setlength{\parindent}{0pt}
\setlength{\parskip}{6pt}
\onehalfspacing

\begin{document}

%----------------------------------
% Title Page (English, NUJ style with logo)
%----------------------------------
\begin{titlepage}
\centering
\vspace*{0.5cm}

% NUJ logo from official website
\IfFileExists{nuj-logo.png}{\includegraphics[width=4cm]{nuj-logo.png}\\[0.5cm]}{}
\IfFileExists{nuj-logo.pdf}{\includegraphics[width=4cm]{nuj-logo.pdf}\\[0.5cm]}{}

{\Large\bfseries NIRWAN UNIVERSITY JAIPUR}\\[0.5cm]

\vspace{0.8cm}

{\Large\bfseries Cognitive Governance:\\[4pt]
The Role of Executive Decision Hygiene in Modulating AI-Augmented Strategy Under Uncertainty\par}

\vspace{1.8cm}

{\large A Synopsis submitted to Nirwan University Jaipur in partial fulfillment of the requirement for the degree of\\[4pt]
Doctor of Philosophy\par}

\vspace{0.8cm}

{\large in the subject of\\[4pt]
\textbf{Management}\par}

\vspace{1.4cm}

{\large by\\[4pt]
\textbf{Mr. Syed Faraaz}\\[4pt]
Enrolment No.: \underline{\hspace{4cm}}\par}

\vfill

{\large Under the Supervision of\\[4pt]
\textbf{Dr. \underline{\hspace{4cm}}}\\[2pt]
School of \underline{\hspace{4cm}}\par}

\vspace{1.8cm}

{\large NIRWAN UNIVERSITY\\[4pt]
JAIPUR, RAJASTHAN\par}

\vspace*{0.8cm}
\end{titlepage}

%----------------------------------
% Research Proposal / Synopsis (NUJ outline)
%----------------------------------
\section*{Research Proposal / Synopsis\\[2pt]\normalsize (Outline of Proposed Research Work)}

\subsection*{1. Introduction}

Founders and top managers increasingly rely on predictive tools and dashboards to support strategic decisions -- for example, whether to pivot a business model, expand into a new market, or adjust hiring and burn. Systems such as the CAMP/Flash framework produce structured assessments of Capital, Advantage, Market, and People, along with probability-style forecasts of venture success based on thousands of assessed startups.

At the same time, decades of work in behavioural decision making show that managerial judgements are subject to bias, noise, and bounded rationality, especially under pressure and uncertainty. Kahneman, Sibony and Sunstein (2021) have argued that ``decision hygiene'' -- simple procedures such as independent pre-judgements, structured discussions, and reference class checks -- can reduce unwanted variability and bias in high-stakes judgements.

In this project, the candidate leverages direct access to the operational Flash CAMP system, which contains a rich, structured dataset on several thousand venture-backed startups and their outcomes. This access makes it possible to study high-stakes decisions not only in controlled scenarios but also in real ventures that use predictive tools in practice.

When algorithmic forecasts enter the room, founders face a new leadership task: they must balance model output with their own experience and with signals from their teams. Poorly structured processes can lead to over-reliance on models (automation bias) or to defensive rejection of useful warnings. Both are dangerous for young firms with limited capital and time.

This study therefore focuses on \textbf{executive decision hygiene} as a practical, people-centric way to structure high-stakes decisions when algorithmic forecasts are available. The concern is not the technical design of the model, but the leadership and governance of the decision process: who speaks when, how disagreement is handled, and how the final choice is justified. The work is situated within Management, combining behavioural strategy, leadership and entrepreneurship, with a specific emphasis on early-stage ventures that already use the Flash CAMP system.

\subsection*{2. Review of Related Literature}

The proposed work draws on four main bodies of literature:

\textbf{Strategic decision making and behavioural strategy.}
Classic strategy studies highlight how top managers make complex, non-routine decisions under time pressure and ambiguity. Behavioural strategy adds insights from psychology, showing how heuristics, biases and limited attention shape strategic choices.

\textbf{Decision hygiene and noise reduction.}
Recent work on ``noise'' in judgement emphasises that different decision makers, or the same decision maker at different times, may give very different answers to the same problem. Decision hygiene has been proposed as a way to improve consistency and quality through structured procedures. However, this has not been studied in founder-level strategic decisions, particularly when model outputs are involved.

\textbf{Decision support systems and algorithmic advice.}
Research on decision support and algorithmic advice shows that people can exhibit automation bias (over-trusting systems) or algorithm aversion (under-using them after a failure). Much of this evidence comes from operational or highly standardised tasks (credit scoring, medical diagnosis), not from strategic decisions in young firms.

\textbf{Leadership, psychological safety and voice.}
Organisational behaviour research indicates that psychological safety -- the shared belief that it is safe to speak up -- strongly influences learning, error detection and team performance. Leaders shape this climate by how they run meetings, respond to dissent and explain decisions. The effect of algorithmic forecasts on safety and voice in leadership teams is largely unexplored.

In the entrepreneurship and startup governance literature, numerous studies examine how founders and investors structure financing, boards and experiments. Very few, however, consider how predictive tools are integrated into these decisions and how the human process around such tools affects both people and outcomes. The availability of the Flash CAMP dataset and system provides an unusual opportunity to connect these strands using real assessment and outcome data.

\subsection*{3. Research Gap}

From the literature, the following gaps are evident:

\begin{itemize}[leftmargin=*]
  \item Most empirical studies of algorithmic advice focus on individual or operational decisions, not on \textbf{founder-level strategic decisions} with long-term consequences.
  \item The idea of \textbf{decision hygiene} is well developed conceptually but rarely measured or observed in real strategic meetings that use model outputs.
  \item Existing work does not explain how decision processes around algorithms affect \textbf{psychological safety and voice} within the founding team.
  \item There is little evidence linking the presence or absence of decision hygiene in AI-supported decisions to actual \textbf{venture outcomes} such as survival, pivot success or capital efficiency, especially using large, structured datasets.
\end{itemize}

This research aims to address these gaps by concentrating on early-stage ventures that use the Flash CAMP system, observing and measuring decision hygiene in episodes where algorithmic forecasts are consulted, and connecting process differences to both team climate and outcomes.

\subsection*{4. Objectives of the Study}

The main objectives are:

\begin{enumerate}[leftmargin=*]
  \item To understand how founders and their top teams structure high-stakes strategic decisions when algorithmic forecasts are available.
  \item To identify and describe the specific decision hygiene practices that are actually used in such decisions.
  \item To examine how these practices influence team climate, particularly psychological safety and voice during decision discussions.
  \item To assess whether decisions taken under structured decision hygiene are associated with better perceived decision quality and, where possible, better venture outcomes.
  \item To make disciplined use of the Flash CAMP dataset to relate decision processes to observed venture trajectories.
  \item To develop practical recommendations for founders, boards and investors on designing decision processes that combine algorithmic forecasts with human judgement in a responsible way.
\end{enumerate}

\subsection*{5. Importance of the Proposed Research Work}

\textbf{Academic importance.}
The study extends behavioural strategy by examining how procedural safeguards (decision hygiene) operate in leadership teams using predictive tools, rather than in isolated individual judgements. It contributes to leadership and organisational behaviour literature by linking decision structure around algorithms to psychological safety and voice. It also adds to entrepreneurship research by connecting decision processes in AI-supported strategic choices to venture performance, using a rare, large-scale dataset from the Flash CAMP system.

\textbf{Practical importance.}
Founders and investors need simple, actionable guidance on how to use dashboards and risk scores without suppressing human insight or team voice. The findings can inform the design of meeting protocols, checklists and training for founders and investment committees. Because the Flash CAMP system is already in production, the research can be translated directly into product features (for example, decision logs and override prompts) and into advisory playbooks for accelerators, venture funds and corporate innovation units.

\subsection*{6. Hypotheses of the Study}

The study will test the following hypotheses (subject to refinement):

\begin{itemize}[leftmargin=*]
  \item \textbf{H1:} Teams using structured decision hygiene (such as independent pre-judgement and explicit dissent) will make more balanced use of algorithmic forecasts (neither automatic acceptance nor automatic rejection) than teams without such structure.
  \item \textbf{H2:} The presence of decision hygiene practices will be positively associated with perceived psychological safety in decision meetings.
  \item \textbf{H3:} In teams using decision hygiene, more members will voice disagreement with the founder or with the model output (higher voice behaviour) than in teams without these practices.
  \item \textbf{H4:} Decisions taken under structured decision hygiene will be rated as higher quality by participants (in terms of fairness, thoroughness and confidence) than decisions taken without such practices.
  \item \textbf{H5 (exploratory):} Ventures that report consistent use of decision hygiene practices in AI-supported strategic decisions will show better performance over time (e.g., fewer clearly poor pivots, more disciplined capital use), controlling for stage and sector, as measured using Flash CAMP archival data.
\end{itemize}

\subsection*{7. Research Methodology}

The study will use a mixed-method design with three phases.

\textbf{Phase I -- Exploratory qualitative study.}
Semi-structured interviews with founders and key team members, and where feasible, observation of decision meetings in 8--10 early-stage ventures that use predictive tools such as Flash CAMP. The goal is to map current decision practices and identify real examples of decision hygiene in AI-augmented strategic decisions. Thematic analysis will be used to derive common patterns.

\textbf{Phase II -- Scenario-based experiment.}
Small teams (for example, founders, managers or advanced students with startup experience) will work through standardised decision cases that include a model-generated forecast derived from Flash CAMP-like outputs. Teams will be randomly assigned to a ``business as usual'' condition or a ``decision hygiene'' condition that includes independent pre-judgement, required expression of dissent and explicit discussion of reasons for following or overriding the model. Measures will include use of the forecast (follow vs override), psychological safety, frequency of voiced dissent, perceived decision quality and decision accuracy where the case has a defined preferred outcome.

\textbf{Phase III -- Field survey and archival analysis.}
A survey of founders and top-team members will be used to construct a Decision Hygiene Index and to measure psychological safety and perceived value of predictive tools. Where possible, survey data will be linked to archival performance indicators (e.g., survival, pivot history, capital efficiency) using internal records and structured assessment data from the Flash CAMP system. Regression and, where appropriate, survival or panel models will be used to test the association between decision hygiene and performance, controlling for venture age, stage, size and sector.

All data collection will follow ethical guidelines, with informed consent, anonymisation of sensitive information and appropriate approvals from the university.

\subsection*{8. References / Bibliography (Illustrative)}

\begin{itemize}[leftmargin=*]
  \item Cyert, R. M., \& March, J. G. (1963). \emph{A Behavioral Theory of the Firm}. Englewood Cliffs, NJ: Prentice-Hall.
  \item Edmondson, A. C. (1999). Psychological safety and learning behavior in work teams. \emph{Administrative Science Quarterly}, 44(2), 350--383.
  \item Eisenhardt, K. M. (1989). Making fast strategic decisions in high-velocity environments. \emph{Academy of Management Journal}, 32(3), 543--576.
  \item Kahneman, D., Sibony, O., \& Sunstein, C. R. (2021). \emph{Noise: A Flaw in Human Judgment}. New York: Little, Brown Spark.
  \item Milkman, K. L., Chugh, D., \& Bazerman, M. H. (2009). How can decision making be improved? \emph{Perspectives on Psychological Science}, 4(4), 379--383.
  \item Simon, H. A. (1979). Rational decision making in business organizations. \emph{American Economic Review}, 69(4), 493--513.
\end{itemize}

\end{document}

